% !TEX root = ../root.tex

\section{Finite State Automata}
DFA, NFA, RE. Lexical analysis, network protocol specs, string searching. Geometric (FA) vs algebraic (RE) descriptions; spaghetti vs structured code.
\label{sec:moore}

\subsection{State Machines in Scala}
\firstthought{There are many ways to implement} the state machine concept in code. The essence is that
the input is processed one item at a time, in order, with only a fixed amount of
information (the ``state'') preserved from one item to the next. The iteration over the
items may be performed by a loop or recursion; the state may be maintained explicitly
in variables or hidden in an object, or it may be implicit in the current section of
code being executed; the transition from one state to the next may be controlled by a
series of conditional statements, a data structure representing the transition graph,
or a function which encapsulates the required logic. Here are some examples:

\paragraph{Strings containing the vowels in order}

\firstthought{This is the example} from Section~10.2 of Aho \& Ullman. A string of lower-case letters
will be accepted if it contains the vowels \verb|a|, \verb|e|, \verb|i|, \verb|o|, and \verb|u|, in that order
(the vowels may occur in other positions as well, as in \verb|"sacrilegious"|). For example, it should accept \verb|"abstemious"| and \verb|"aeiou"|, but not \verb|"religious"| (there is no \verb|a|) or \verb|"undercoating"| (the vowels are in the wrong order). Here is a diagram of the state machine (Figure~10.3 of Aho \& Ullman):
\begin{center}
\begin{tikzpicture}[->,>=stealth',shorten >=1pt,auto,node distance=1.5cm,semithick]
  \node[initial,state](q0){0};
  \node[state](q1)[right of=q0]{1};
  \node[state](q2)[right of=q1]{2};
  \node[state](q3)[right of=q2]{3};
  \node[state](q4)[right of=q3]{4};
  \node[accepting,state](q5)[right of=q4]{5};
  
  \path(q0) edge [loop above] node {not \texttt{a}} (q0);
  \path(q1) edge [loop above] node {not \texttt{e}} (q1);
  \path(q2) edge [loop above] node {not \texttt{i}} (q2);
  \path(q3) edge [loop above] node {not \texttt{o}} (q3);
  \path(q4) edge [loop above] node {not \texttt{u}} (q4);
  \path(q0) edge node {\texttt{a}} (q1);
  \path(q1) edge node {\texttt{e}} (q2);
  \path(q2) edge node {\texttt{i}} (q3);
  \path(q3) edge node {\texttt{o}} (q4);
  \path(q4) edge node {\texttt{u}} (q5);
\end{tikzpicture}
\end{center}

\newthought{Using state implicit in conditional statements:}
Figure~\ref{fig:vowelsimplicit} gives a Scala version of Figure~10.2 from Aho \& Ullman. Calling the \verb|testWord| function should return \verb|true| if its argument is accepted, and \verb|false| if not.
\begin{figure}
\begin{verbatim}
def testWord(word: String): Boolean = {
  var index = 0
  
  /**
   * Advance index through the characters of word until either
   * c is found (return true; index points at character after c), or
   * the end of the string is reached (return false).
   */
  def findChar(c: Char): Boolean = {
    while (index < word.length && word(index) != c) {
      index += 1
    }
    
    if (index == word.length) {
      false
    } else {
      index += 1
      true
    }
  }
  
  // state 0
  if (findChar('a'))
    // state 1
    if (findChar('e'))
      // state 2
      if (findChar('i'))
        // state 3
        if (findChar('o'))
          // state 4
          if (findChar('u'))
            // state 5
            return true
  // error state
  false
}
\end{verbatim}
\caption{Scala code to search for vowels in order using state implicit in conditionals}
\label{fig:vowelsimplicit}
\end{figure}

The input is processed in this example by the \verb|while| loop in the \verb|findChar|
function; the variable \verb|index| is used to step through the characters in the word.
The current ``state'' is reflected in how far the execution has progressed through
the nested \verb|if| statements in \verb|testWord|; after reading characters for a while in
state 0, it transitions to state 1 when the first \verb|a| is seen, then to state 2 upon
seeing a following \verb|e|, \textit{etc}. If any of the calls to \verb|findChar| return \verb|false|, meaning
the end of the string has been reached while looking for one of the vowels, then the
machine enters an error state. If all of the calls to \verb|findChar| succeed, then the
machine reaches state 5 and immediately returns \verb|true| (without looking at the rest
of the string).

\newthought{Using integer state with transitions in a graph:}
Figure~\ref{fig:vowelsinteger} is the same state machine, with the state explicitly represented by an integer
in the range 0 to 5. The transitions are stored in an array of maps: \verb|trans(i)| is the
map, for state \verb|i|, from the current character to the next state. Each of the maps in
this case is particularly simple, since at most one edge leads away from each state
to another; Scala allows a map to have a ``default value,'' so that any transition not
explicitly specified will go to that default state (here, remaining in the same
state). See below for other examples using more complicated graphs.
\begin{figure}
\begin{verbatim}
val trans = Array[Map[Char, Int]](
  Map('a' -> 1) withDefaultValue 0, // state 0
  Map('e' -> 2) withDefaultValue 1, // state 1
  Map('i' -> 3) withDefaultValue 2, // state 2
  Map('o' -> 4) withDefaultValue 3, // state 3
  Map('u' -> 5) withDefaultValue 4, // state 4
  Map() withDefaultValue 5 // state 5
)

def testWord(word: String): Boolean = {
  var state = 0 // initial state
  for (c <- word) {
    state = trans(state)(c)
  }
  state == 5 // returns true if accepting state reached
}
\end{verbatim}
\caption{Scala code to search for vowels in order using integer state and maps}
\label{fig:vowelsinteger}
\end{figure}

\newthought{Using object-oriented state and transitions:}
Instead of assigning arbitrary numbers to the states, and collecting all of the
transition information into a global graph data structure, Figure~\ref{fig:vowelsobjects} shows a more object-oriented
approach that associates an object with each state.
\begin{figure}
\begin{verbatim}
object testWord {
  trait State {
    def trans(c: Char): State
    def accept: Boolean = false // override to identify final state
  }

  object InitialState extends State {
    def trans(c: Char) = if (c == 'a') AState else InitialState
  }

  object AState extends State {
    def trans(c: Char) = if (c == 'e') AEState else AState
  }

  object AEState extends State {
    def trans(c: Char) = if (c == 'i') AEIState else AEState
  }

  object AEIState extends State {
    def trans(c: Char) = if (c == 'o') AEIOState else AEIState
  }

  object AEIOState extends State {
    def trans(c: Char) = if (c == 'u') AEIOUState else AEIOState
  }

  object AEIOUState extends State {
    def trans(c: Char) = AEIOUState
    override def accept = true
  }

  def apply(word: String): Boolean = {
    var state: State = InitialState
    for (c <- word) {
      state = state.trans(c)
    }
    state.accept
  }
}
\end{verbatim}
\caption{Scala code to search for vowels in order using object-oriented state}
\label{fig:vowelsobjects}
\end{figure}

\paragraph{Vending Machine}

\firstthought{Suppose we want to model} a vending machine which accepts nickels, dimes, and quarters,
and dispenses a piece of candy when 25 cents has been deposited. If more than 25 cents
is put in (for example, three dimes), then after dispensing the candy the remaining
amount is applied toward the next transaction (that is, it doesn't give any change).
One difference in this machine is that each edge in the graph will
give not only a new state but also an indication of whether candy was given out on
the transition (technically, this is known as a ``Mealy Machine''). There is no accepting state, since the machine will keep
running as long as there is input; in the code below, we will print \verb|"Candy!"|
whenever it produces a piece of candy. For example, when the input is \verb|"NNNNQDNND"|,
the output should be to print \verb|"Candy!"| three times. Here is the state diagram, where a transition labelled \texttt{c}/\0 means the input is coin \texttt{c} and no candy is output, while \texttt{c}/\1 means you get candy:
\begin{center}
\begin{tikzpicture}[->,>=stealth',shorten >=1pt,auto,semithick]
  \node[initial,state](q0){0\textcent};
  \node[state](q5)[above right=1cm and 2cm of q0]{5\textcent};
  \node[state](q10)[right=4cm of q0]{10\textcent};
  \node[state](q15)[below right=1.7cm and 3.3cm of q0]{15\textcent};
  \node[state](q20)[below right=1.7cm and 0.7cm of q0]{20\textcent};
  
  \path(q0) edge node [sloped,pos=0.7] {\texttt{N}/\0} (q5)
            edge node [sloped,pos=0.25] {\texttt{D}/\0} (q10)
            edge [loop above] node {\texttt{Q}/\1} (q0);
  \path(q5) edge node [sloped,pos=0.3] {\texttt{N}/\0} (q10)
            edge node [sloped,pos=0.15] {\texttt{D}/\0} (q15)
            edge [loop above] node {\texttt{Q}/\1} (q5);
  \path(q10) edge node [sloped,pos=0.7] {\texttt{N}/\0} (q15)
            edge node [sloped,pos=0.3] {\texttt{D}/\0} (q20)
            edge [loop above] node {\texttt{Q}/\1} (q10);
  \path(q15) edge node [sloped] {\texttt{N}/\0} (q20)
            edge node [sloped,pos=0.1] {\texttt{D}/\1} (q0)
            edge [loop below] node {\texttt{Q}/\1} (q15);
  \path(q20) edge node [sloped,pos=0.3] {\texttt{N}/\1} (q0)
            edge node [sloped,pos=0.3] {\texttt{D}/\1} (q5)
            edge [loop below] node {\texttt{Q}/\1} (q20);
\end{tikzpicture}
\end{center}

\newthought{Using integer state with transitions in a graph:}
We will adopt a similar solution as for the second version of the vowel problem. This
time, the integer state numbers will be more meaningful: they will be 0, 5, 10, 15,
and 20, representing the amount of money which has been deposited so far. Since these
are not consecutive, the graph will be represented by a map of maps instead of an
array of maps. Also, there will be no need for default transitions, since each of the
three possible inputs (\verb|N|, \verb|D|, or \verb|Q|, for nickel, dime, and quarter) will cause a
different change of state. Finally, the value of the map for each input will be a pair of the next state
and a Boolean indicating whether a piece of candy should be output. The code is shown in Figure~\ref{fig:vendinginteger}.
\begin{figure}
\begin{verbatim}
val vtrans = Map[Int, Map[Char, (Int, Boolean)]](
   0 -> Map('N' ->  (5, false),
            'D' -> (10, false),
            'Q' -> (0, true)),
   5 -> Map('N' -> (10, false),
            'D' -> (15, false),
            'Q' -> (5, true)),
  10 -> Map('N' -> (15, false),
            'D' -> (20, false),
            'Q' -> (10, true)),
  15 -> Map('N' -> (20, false),
            'D' ->  (0, true), 
            'Q' -> (15, true)),
  20 -> Map('N' ->  (0, true),
            'D' ->  (5, true),
            'Q' -> (20, true))
)

def vend(input: String) {
  var state = 0 // initial state
  for (coin <- input) {
    val (newState, candy) = vtrans(state)(coin)
    if (candy) println("Candy!")
    state = newState
  }
}
\end{verbatim}
\caption{Scala code to simulate a vending machine using integer state and maps}
\label{fig:vendinginteger}
\end{figure}

\newthought{Using an object-oriented approach with encapsulated state:}
Instead of exposing an explicit state, we can wrap it up inside an object with
(mutable) internal state. Figure~\ref{fig:vendingobject} is another approach to the vending machine, which also
replaces the discrete transition graph with a calculated transition function.
\begin{figure}
\begin{verbatim}
class VendingMachine(price: Int) {
  private var balance = 0
  
  /**
   * Insert the given amount of money;
   * returns true if an item is vended.
   */
  def deposit(amount: Int): Boolean = {
    balance += amount
    if (balance >= price) {
      balance -= price
      true
    } else {
      false
    }
  }
}

def vend(input: String) {
  val machine = new VendingMachine(25)
  for (coin <- input) {
    val candy = coin match {
      case 'N' => machine.deposit(5)
      case 'D' => machine.deposit(10)
      case 'Q' => machine.deposit(25)
    }
    if (candy) println("Candy!")
  }
}
\end{verbatim}
\caption{Scala code to simulate a vending machine using object-oriented state}
\label{fig:vendingobject}
\end{figure}
